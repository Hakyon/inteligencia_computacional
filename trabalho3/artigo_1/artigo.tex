 % =======================================================================
% =                                                                     =
% =======================================================================
% -----------------------------------------------------------------------
% - Author:     Chaua Queirolo                                          -
% - Version:    001                                                     -
% -----------------------------------------------------------------------
\documentclass[a4paper,11pt]{article}    

% =======================================================================
% PACKAGES
% =======================================================================

% Language support
\usepackage[brazil]{babel}
\usepackage[utf8]{inputenc}
\usepackage[T1]{fontenc}
\usepackage{ae,aecompl}
\usepackage[section]{placeins}



% Configuration
\usepackage{url}
\usepackage{enumerate}
\usepackage{color}
\usepackage[svgnames,table]{xcolor}
\usepackage[margin=2cm,includefoot]{geometry}

% Tabular
\usepackage{multirow}
\usepackage{multicol}

% Images
\usepackage{graphicx}
\usepackage[scriptsize]{subfigure}
\usepackage{epstopdf}
\usepackage{float}% http://ctan.org/pkg/{multicol,lipsum,graphicx,float}


% Math
\usepackage{mdwtab}	% bug rowcolor
\usepackage{amssymb}
\usepackage{amsmath}
\usepackage{footnote}


\usepackage{enumitem}

% References
\usepackage[sort,nocompress]{cite}

% =======================================================================
% VARIABLES
% =======================================================================

% Space between the lines in a table
\renewcommand{\arraystretch}{1.3}

% Define a new column type
\newcolumntype{x}[1]{>{\raggedright\hspace{0pt}}p{#1}}%

% Controle das Margens
\sloppy
\tolerance=9999999

% Espaço entre colunas
\setlength{\columnsep}{.9cm}


% Configuration
\usepackage{lipsum}
\usepackage{blindtext}

% =======================================================================
% HEADER
% =======================================================================

\title{Computação Evolutiva}
\author{Alan Azevedo Bancks \\ Universidade Tuiuti do Parana \\
Inteligência Computacional \\E-mail: {\tt dsalan@hotmail.com}}
\date{18/08/2019}

\newenvironment{Figure}
  {\par\medskip\noindent\minipage{\linewidth}}
    {\endminipage\par\medskip}

% =======================================================================
% DOCUMENT
% =======================================================================
\begin{document}

\maketitle

%\begin{abstract}
%    \lipsum[1]
%\end{abstract}
%\hspace{.5cm}

\begin{multicols}{2}
\section{Introdução}

A computação evolutiva usa técnicas em seus algoritmos que funcionam de maneira semelhante ao mecanismo biologico 
da propria evolução proposto por Charles Darwin, onde a sobrevivencia do mais apto tendo caracteristicas
mais adaptativas é o que lhe garante passar seus genes adiante, e a sua herança dispoem de uma geração mais evoluida
de maneira gradual com o passar dessas gerações.
Para que um problema possa ser passivo do uso desse tipo de técnica , ele precisa ser adpatado a esses conceitos biologicos, 
com isso garantido que a estrutura do algoritmo seja funcional e traga resultados beneficos.\cite{artigo_motion} 


%\section{Revisão Bibliografica}



\section{Aplicações}

As soluções dadas para problemas onde são relevantes essas técnicas podem ser diversas , devido a variabilidade das caracteristicas
que o problema assume em cada geração;Otimazação é um bom alvo a ser conquistado pois os Algoritmos Genéticos são bons para problemas 
com muitas variaveis e espaços de soluções elevados. Exemplos: Roteamento de redes, otimização de trajetos , gerenciamento de filas 


\section{Métodologias}


\subsection{O Algoritmo Genético}


Imcorpotando o modelo biologico , uma estrutura é construida como um cromossomo e caracteristicas vitais a solução do problema 
são mantidadas conforma gerações de evolução vão avançando. Normalmente possuem descrições de entradas formadas de cadeias de bits
de tamanho fixo , existem 3 tipos de representações possiveis de cromosso , a binaria, inteira e real






\section{Conclusão}


\end{multicols}
\bibliographystyle{plain}
\bibliography{referencias}

\end{document}
