 % =======================================================================
% =                                                                     =
% =======================================================================
% -----------------------------------------------------------------------
% - Author:     Chaua Queirolo                                          -
% - Version:    001                                                     -
% -----------------------------------------------------------------------
\documentclass[a4paper,11pt]{article}    

% =======================================================================
% PACKAGES
% =======================================================================

% Language support
\usepackage[brazil]{babel}
\usepackage[utf8]{inputenc}
\usepackage[T1]{fontenc}
\usepackage{ae,aecompl}
\usepackage[section]{placeins}



% Configuration
\usepackage{url}
\usepackage{enumerate}
\usepackage{color}
\usepackage[svgnames,table]{xcolor}
\usepackage[margin=2cm,includefoot]{geometry}

% Tabular
\usepackage{multirow}
\usepackage{multicol}

% Images
\usepackage{graphicx}
\usepackage[scriptsize]{subfigure}
\usepackage{epstopdf}
\usepackage{float}% http://ctan.org/pkg/{multicol,lipsum,graphicx,float}


% Math
\usepackage{mdwtab}	% bug rowcolor
\usepackage{amssymb}
\usepackage{amsmath}
\usepackage{footnote}

% References
\usepackage[sort,nocompress]{cite}

% =======================================================================
% VARIABLES
% =======================================================================

% Space between the lines in a table
\renewcommand{\arraystretch}{1.3}

% Define a new column type
\newcolumntype{x}[1]{>{\raggedright\hspace{0pt}}p{#1}}%

% Controle das Margens
\sloppy
\tolerance=9999999

% Espaço entre colunas
\setlength{\columnsep}{.9cm}


% Configuration
\usepackage{lipsum}
\usepackage{blindtext}

% =======================================================================
% HEADER
% =======================================================================

\title{Resenha: APLICAÇÃO DE REDES NEURAIS ARTIFICIAIS COMO TESTE DE DETECÇÃO DE FRAUDE DE LEITE POR ADIÇÃO DE SORO DE QUEIJO}
\author{Alan Azevedo Bancks \\ Universidade Tuiuti do Parana \\E-mail: {\tt dsalan@hotmail.com}}
\date{03/05/2018}

\newenvironment{Figure}
  {\par\medskip\noindent\minipage{\linewidth}}
    {\endminipage\par\medskip}

% =======================================================================
% DOCUMENT
% =======================================================================
\begin{document}

\maketitle

%\begin{abstract}
%    \lipsum[1]
%\end{abstract}
%\hspace{.5cm}

\begin{multicols}{2}
\section{Resenha}
A adulteração do leite com a adição de soro de leite problema recorrente, anteriormente era adicionado água e desnate para produção de creme de leite, depois surgiram novos tipos de alterações como adição de soro de queijo, conservantes, neutralizantes e reconstituintes da densidade e crioscopia (sal, açucar e amido) (Abrantes, Campelo \& Silva, 2014).
A adição de soro de queijo advém da facilidade de produção e baixo custo, pois, em média na fabricação de um jeito é gerado de 9 à 12 litros de soro. A adulteração também é reforçada pelo fato que o soro é um liquido difícil de ser usado para produção de outros alimentos. A fraude é detectada pela quantidade de  caseinomacropeptídeo (CMP), fragmento que no processo comum do leite não é detectável, porém no do soro permanece ou seja deve estar ausente no leite, porém o custo para realizar essa detecção é elevado, devido aos equipamentos e profissionais capacitados, e mesmo assim pode ocorrer falhas no processo, as análises de rotina normalmente só dectam a presença do soro quanto está em grandes quantias.

Para realizar a avaliação e interpretação dos dados de rotinas alguns pesquisadores têm analisado o potencial de aplicação da estatística multivariada (MANOVA), especificamente da análise discriminante. Emerge na Redes Neurais Artificiais (RNA) uma possibilidade para classificação de amostras.  As RNA podem funcionar como modelos preditivos que descrevem a relação funcional entre as variáveis de entrada e variáveis de saída de um sistema. Possuindo várias vantagens sobre os modelos fenomenológicos tradicionais ou modelos empíricos. RNA desenvolvem um mapeamento das variáveis de entrada e saída, que podem ser usados para predizer parâmetros de saída do sistema (SINGH et al., 2009 apud Valente, Guimares, Gaspardi \& Oliveira, 2014).

Conforme pesquisa apresentada por Valente et al. (2014), a utilização de rede neurais artificiais permite a detecção de fraude por adição de soro de queijo ao leite a partir dos resultados obtidos logo na análise de rotina, o processo de pesquisa foi realizado em uma fazenda escola de Minas,  como critério de seleção para as redes foi usado o erro quadrático médio (EQM) para os dados de validação (CARVALHO et al., 2013 apud Valente, Guimares, Gaspardi \& Oliveira, 2014). As variáveis de entrada foram: temperatura, pH, teor de gordura, extrato seco desengordurado, proteínas, ponto de congelamento, condutividade, lactose e densidade das amostras, para camada de saída foram testados duas alterações : uma com um neurônio em que leite normal era 0 (zero) e adulterado era 1 (um); outra com dois neurônios, leite normal (0, 1) e leite adulterado (1, 0). 

O EMQ apresentou se válido e das 28.017 redes testadas, entre Multilayer perceptron e RBF, a melhor rede foi a que apresentou menor número de erros de classificação, menor diferença entre os erros de classificação para treinamento, validação e teste, e como último critério, o menor número de neurônios na camada oculta. Na camada de saída as redes que obtiveram o melhor resultado foi a  com dois neurônios. 
A neural de função de base radial foi a melhor para classificação e com melhor resultado apresentou 10 neurônios na camada de entrada, 40 neurônios na camada oculta e 2 na camada de saída.
Conforme dados obtidos a RNA apresenta potencial para ser utilizada nas avaliações de hipóteses de fraude do leite, pois apresenta a possibilidade de reduzir se o número de amostras que iriam ser analisadas pela metodologia tradicional







\bibliographystyle{plain}
\bibliography{referencias}

\end{multicols}
\end{document}

