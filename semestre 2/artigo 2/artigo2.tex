% Todo documento é iniciado pelo \documentclass
\documentclass[11pt,twocolumn]{article}

\usepackage[brazil]{babel}
\usepackage[utf8]{inputenc}
\usepackage[T1]{fontenc}

\usepackage{graphicx}
\usepackage[caption=false]{subfig}

% Definição das variáveis do texto
\title{Sistema de algoritmo que determina pena de condenados cria polêmica nos EUA}
\author{Alan Azevedo Bancks}
%\date{}

% Ambiente document: onde fica o conteúdo do texto
\begin{document}

% Criar o título do documento
\maketitle

|% \tableofcontents


% O documento é organizado em seções
% \part -> book
% \chapter{capitulo} -> book / report
% \section{seção}
% \subsection{subseção}
% \subsubsection{subsubseção}

\section{Introdução}


O uso da inteligencia Artificial na tomada de decisões que impactam cada vez mais nossas vidas tem crescido 
de forma acelerada e para que seja tomada uma boa decisão segundo  ,quanto maior o volume e informações analisadas, melhor 
a eficacia do resultado.Existem diversos processos no dia-a-dia de governos e corporações que acabam por serem muito burocráticas e lentas devido ao grande volume de informações que devem ser analisadas, fora a quantidade de pessoas e suas limitações de carga horaria de trabalho diaria.No EUA, em Wisconsin foi implantado uma solução utilizando IA para analisar o grau de peliculosidade de um criminoso que por sua vez 
influencia fortemente na decisão da pena de quem esta sendo analisada e esse tipo de ferramenta tem causado certa polemica por conta
das sentenças que vem sendo tomadas. O algoritmo ultilizado no desenvolvimento dessa IA, toma como base o Compas (sigla em inglês para Correctional Offender Management Profiling for Alternative Sanctions


\section{Analise critíca}

O algoritmo por traz da IA é segredo corporativo e portando não se sabe exatamente de que forma ele julga os valores analisados como entrada,
apenas certas características , mas observando seus resultados foi possivel notar diferenças na forma como são julgados as pessoas de acordo 
com sua etnia 


\section{Referencias}

\bibliographystyle{plain}
%\bibliography{referencias}


\end{document}
