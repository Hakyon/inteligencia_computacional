% Todo documento é iniciado pelo \documentclass
\documentclass[11pt,twocolumn]{article}

\usepackage[brazil]{babel}
\usepackage[utf8]{inputenc}
\usepackage[T1]{fontenc}

\usepackage{graphicx}
\usepackage[caption=false]{subfig}

% Definição das variáveis do texto
\title{Sistema de algoritmo que determina pena de condenados cria polêmica nos EUA}
\author{Alan Azevedo Bancks}
%\date{}

% Ambiente document: onde fica o conteúdo do texto
\begin{document}

% Criar o título do documento
\maketitle

|% \tableofcontents


% O documento é organizado em seções
% \part -> book
% \chapter{capitulo} -> book / report
% \section{seção}
% \subsection{subseção}
% \subsubsection{subsubseção}

\section{Resenha}


A utilização da Inteligência Artificial no recrutamento e seleção de pessoas se tornou alvo de interesse das organizações, porém o tema gera opiniões diversas.
Os dados indicam o notável preconceito nos processos, contratações que priorizam raça, idade, gênero, etc. Conforme Alan Todd, CEO da CorpU, “As pessoas escolhem quem gostam com base em preconceitos inconscientes. ”

A matéria expressa que essa seleção desigual vem da estruturação dos processos e das pessoas que recrutam, podemos refletir; até que pontoç se de fato querem....
Enquanto alguns argumentam que a IA é capaz de eliminar o preconceito, existe a dificuldade de manter a IA com uma triagem de equidade, visto que ela pode ter influência da opinião humana em seu treinamento.
[Refazer : Falar sua opinião sobre a falta de padrão no desenvolvimento dos algoritmos.]


Algoritmos não supervisionados mantem seus filtros por padrões determinados, não sendo capazes de descobrir perfis em potencial. Para amenizar essa questão, pode ser realizado uma auditoria dos algoritmos, com isso os programadores são capazes de identificar aprendizagens preconceituosas da IA e corrigi-la.
O texto apresenta diversas opiniões, favorecendo ou desconfiando da capacidade das inteligências artificiais e sua utilização.
É questionável se essa implementação de fato retiraria todo o preconceito no processo, usar uma IA para preencher candidatos de certos grupos pode não ser favorável, pois, a máquina levaria em consideração perfis com características técnicas ou comportamentais selecionadas, sem considerar aspectos comportamentais e emocionais intrínsecos. Para Borocas a visão da IA como revolvedora do problema é utópica, “Não deveríamos achar que é uma solução milagrosa”.
O uso da IA em conjunto final com humanos poderia trazer mais ganhos significativos e assim não teríamos uma seleção tão "fria". 


\section{Referencias}

%\bibliographystyle{plain}
%\bibliography{referencias}


\end{document}
